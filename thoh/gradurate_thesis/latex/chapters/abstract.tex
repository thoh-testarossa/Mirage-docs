\begin{abstract}
时至今日,得益于并行计算领域取得的包括通用并行计算平台编程(例如GPGPU),新兴的的并行硬件(例如FPGA)的开发等巨大成就,许多的应用程序抑或是系统,都采用了并行计算的方式来优化他们的性能,这其中也包括了数据库系统。前人在相关领域的工作中,有许多令人瞩目的成果都集中在基于并行计算的查询优化上,例如负载均衡,流水线/并发查询处理模型,等等。得益于这些工作,许多的计算密集型任务的性能来看得到了提升。但是,只有少数的工作将注意力放在数据立方体,以及与之相对应的OLAP计算上。而包括数据立方体生成在内的许多数据立方体相关任务,都属于计算密集型任务,很适合使用并行计算方式来进行优化。

为了解决上述问题,本文提出了一套并行计算模型的方案来提高数据立方体任务,例如底层cuboid的生成,从cuboid到cuboid的计算(用以避免每次从原始数据集计算cuboid的繁重计算)等的执行效率。在我们的工作中,我们通过将已有的算法根据并行平台的特点重新定制,改进了现存的数据立方体的优化算法。在此基础上,我们还提出了一套新的评估函数,以及对应于这套评估函数的优化方案,使其能够适应于不同的硬件配置与不同的计算任务特性。实验结果表明我们的并行计算方案比起同等情况下的非并行计算方案普遍有了2-4倍性能的提升。在此基础之上,我们提出的优化技术又使得原并行算法在数据立方体的计算中加速了15-30\%。

\keywords{中国科学技术大学\zhspace{} 学位论文\zhspace{} 学士\zhspace{} OpenCL\zhspace{} 在线分析处理\zhspace{} 并行计算\zhspace{}
}
\end{abstract}

\begin{enabstract}
Nowadays many applications / systems, including databases, have been used parallel computing technique to optimize their performance because of the great achievements of the parallel computing field such as GPGPU libraries, emerging hardware such as FPGA, GPU and so on. There are many remarkable works which centered on query processing with parallel computing such as workload balancing, pipeline / concurrent query processing model. With their help, as a result, many computation intensive tasks' performance have been improved. However, few of them focus on data cube as well as their corresponding OLAP computation. Actually many data cube-related computation tasks, including data cube generation, are computation intensive that we can use some approaches of parallel computing to optimize them.

To address the problem, this paper provides a parallel computation model to improve the efficiency of data cube tasks such as bottom cuboid generation, inter-cuboid calculation (used to avoid heavy computation which have to face when generating cuboid directly from raw dataset). In our work, we improve existing data cube optimization approaches by customizing them to the parallel settings. Based on that, we provide a new evaluation function and corresponding optimizations, which are adaptable to different hardware configurations and different natures of computational tasks. Experiments show that our prototype is 2-4x better than its non-parallelized counterpart, and our proposed optimization techniques further accelerate the data cube computations efficiency by 15-30\%.

\enkeywords{University of Science and Technology of China (USTC), Thesis, Bachelor, OpenCL, OLAP, Parallel computing}
\end{enabstract}
