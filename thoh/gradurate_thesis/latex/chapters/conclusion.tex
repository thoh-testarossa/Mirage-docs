\chapter{结论}
\citestyle{ustcnumerical}
本文提供了基于OpenCL的利用并行计算平台来对OLAP中数据聚合立方体的生成/计算进行优化的基本动机,设计方案,具体实现和评估结果。事实证明,采用了并行计算的方案在各方面都强于非并行方案,并且基于并行计算平台的特点而进行的一些优化也体现出了一定的效果。

未来该项研究的扩展方向将会往数个方向进行:重新审视花费函数,更加紧密结合并行计算平台的特点,提出更为通用的模型;将其扩展到CPU-GPU结合,或者是其他并行计算硬件的组合的异构计算平台;以及将以上这些技术整合成为一个更为完整的数据库系统。