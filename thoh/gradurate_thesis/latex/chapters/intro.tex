\chapter{简介}
在现代计算机系统及其应用中,数据量的增大已经成为了一个必然的趋势。伴随其而来的计算量的增加,使得越来越多高性能相关的新兴硬件/软件技术的出现成为势在必行。

在高性能计算领域之中,并行计算已经成为了一个很重要的工具。各种新兴硬件的出现,以及伴随它们而诞生的各种并行编程模型,都为高性能计算领域的向前发展提供了十足的力量。

而随着数据规模的增大,对于大量数据的管理、分析也在逐渐成为一项重要的需求,不仅仅是因为本身,更是因为这个领域所做的工作将是所有未来计算机领域的基石。其中作为代表的数据库领域,近些年来不断致力于发展新的数据分析技术,改进数据关联的基本操作与算法,以及更多地尝试新的硬件平台,层次结构,在这些方面都取得了长足的进展。

进入21世纪,人们也开始在各种方面开始尝试将这两者结合在一起,但是由于现行并行计算编程(譬如大计算量,无相关性,弱控制)与传统数据库领域(譬如ACID特性)在某些理念上的冲突,使得在数据库的并行优化方面,还有很多可以开展的工作没有得到人们的充分认识。

在这些工作中,有一个领域是比较重要的:On-Line Analytical Processing(OLAP, 线上分析处理)这个领域的计算中,最主要的部分——数据立方体生成,要求对于用户所需求的数据进行归类、聚合,并且在可控的时间内返回给用户。在短暂的时间内,处理庞大的数据集合,进行复杂的聚合运算,使得这部分成为了一个计算密集的部分。因此,对于这些计算部分,乃至于整个数据立方体聚合操作的性能的优化势在必行。

现有的工作成果中,虽然有很多利用并行计算的方式与特点,对于数据库系统中的其他方面进行优化,从而间接的使得包括数据立方体生成在内的OLAP相关计算的性能有了一定程度的提升,但是,却鲜有工作直接对数据聚合立方体的生成进行优化。

基于上述背景,本文尝试从该领域出发,探寻利用并行计算硬件与编程思想,来尝试从源数据生成数据立方体,以及从数据立方体出发如何生成别的数据立方体这两方面进行优化。
本文的贡献可以从以下几个角度来阐述:首先,本文给出了一套基本的计算代价估计函数,并在这个基础上提出了生成最底层cuboid的时间/空间最优算法,以及结合它们的优点,提出了平衡各自缺点的混合型方法;其次,本文给出了一套如何评估cuboid生成的时间的算法,(并且基于并行计算的基础改进了现有的生成预处理cuboid的方法)。第三,上述的算法均基于OpenCL库实现,为其在多种不同平台间的移植提供了便利。

本文接下来将按照如下方式组织:第2章介绍已有的工作成果与本研究开展的动机,第3章介绍用并行计算从源数据生成底层cuboid的数种方式并分析其优劣,第4章给出一套从源cuboid到目标cuboid的生成方案及其优化,第5章对这个工作进行了评估,第6章将总结本文。
