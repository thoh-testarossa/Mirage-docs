\begin{acknowledgements}

自接触计算机领域以来,我就一直对高性能计算领域抱有浓厚的兴趣。每当看到一个思路敏捷的算法,抑或是一个精妙的实现,我都会认真将其记录下来,在之后细细品读。无论是从算法本身的设计上,还是从如何挖掘新兴的软硬件设施的潜力上,总会有一些厉害的前辈,或者是同辈,甚至可能是后辈们,在不经意间,给予我极大的灵光。

我想,所谓计算机科学的艺术,大概就是这样的吧。

本科的毕业论文,以这个方面的相关研究作为主题,也算是对本科数年,自己寻找的前进方向的一个答复吧。

在论文成文的过程中,我得到了我的导师,中国科学技术大学的谢希科教授的大力支持与帮助。无数次的方向研讨meeting,无数次的交流,这每一分每一秒都包含着老师的心血,包含着老师的智慧。

而说起为何踏进数据库这个领域,选择在数据库这个领域走高性能研究的路子,那是因为有两位至关重要的前辈,充当了我在这个领域的领路人。一位是谢希科教授,另一位则是新加坡国立大学的何炳胜副教授。这两位杰出的前辈,一同向我展示了数据库领域的博大精深,更重要的是,一同向我展现了,一个合格的学术研究者应有的风貌。

同时,我也要感谢在新加坡国立大学实习时的诸位前辈们。是优秀的你们给予了我奋发向上,与你们并肩作战的动力。

我还要感谢无时无刻不在给予我无微不至的关怀的家人们,以及愿意与我分享生活中点点滴滴的喜怒哀乐的朋友们。是你们,让我的生活变得更加安逸舒适,变得更加多姿多彩。

最后,请容许我用一句话,结束这冗长无味但是真挚的致谢,同时拉开新生活的帷幕

\emph{——而那过去了的,就会成为亲切的怀恋}

\end{acknowledgements}
