\chapter{背景工作与研究动机}
\citestyle{ustcnumerical}
\section{Parallel-Database相关}

并行计算是一个发展历史悠久的领域,其最早的发源要追寻到上世纪五十年代后期。上世纪六十年代至七十年代,共享内存的多处理器体系结构出现了。在上世纪八十年代,大规模并行处理器的出现开始占领市场,然而旋即又被集群所取代。发展至今天,并行计算硬件的主流已经成为了多核处理器\cite{ManycoreShift}。

在这个主流下,越来越多的硬件被发掘出来参与到并行计算中来。图形处理器(GPU)最初是被开发来实现以电子显示为主的各方面的图形演算与生成的,然而由于其架构满足流水线深,可利用核心多等特点,很快人们就开始研究GPU的通用用途(GPGPU),其中一个很重要的领域就是GPU参与的并行计算。2006年,NVIDIA首先提出了CUDA,并在2007年首次正式将其发布,这也成为了世界上首个在GPU上进行通用并行编程的平台\cite{cudaofficial}\cite{CUDA}。在不久之后的2009年,苹果公司也由Khronos Group进行开发,提出了自己的一套通用并行计算平台OpenCL\cite{OpenCL}。这一套开发平台不仅仅局限于对GPU的并行潜能的挖掘,而是在此基础上支持了更多的新兴硬件,例如FPGA,DSP等。并行计算在这些优秀平台的支持下开始在各个领域蓬勃发展。

鉴于并行计算其尤其适合对于大量数据的高密度计算的特征,在数据库相关领域,近些年来也有研究者开始了试图将数据库与并行计算硬件(主要是GPU)结合在一起的研究。B. He等人所在的研究组作为最早涉足该领域的研究组之一,在2008-2009年之间出版了相关论文对于关系型数据库在GPU上的可能性进行了探讨。其在2008年发表的论文\cite{heSIGMOD2008}中提到了在GPU上进行的对于关系型数据库join操作的并行优化,而在另一篇2009年的论文中\cite{heACMTDS2009}最早地提出了一套系统的GPU上评估查询执行效率的模型,还有一套完整的在GPU上进行的查询计划的生成方案。这两项研究成果至今仍被相关研究领域的人员频繁引用。另外,2011年发表的另一篇论文\cite{heVLDB2011}中也对传统关系型数据库中的事务处理进行了适合并行架构上的优化。

在接下来的几年,有许多的研究成果基于前述的成果,在并行计算和数据库的结合方面做出了很多的新的尝试。这其中有很多的工作集中在了对于查询的高效处理的方面。俄亥俄州立大学的研究小组在2014年的论文\cite{wangVLDB2014}中首次给出了一种对于并发处理查询的解决方案,改变了从前至多只能并行处理一个委托的不同部分,从而使得总体硬件资源利用效率低下的状况,使得查询之间可以有机会被并行处理,极大地提高了硬件的利用率,提高了查询的吞吐量。南洋理工大学的另一个研究小组\cite{johnsSIGMOD2016}则尝试从另一个角度挖掘并行硬件的架构特点,通过开发一套对于多个查询的流水线处理系统以及与之相协调的并发内核控制和数据交换方法,来使得并行硬件在单位时间内处理多个查询的性能得到了大幅度的提升。另外,本文也尝试着给出了一套查询处理流水线化的分析评估模型。

总的来说,近些年来在数据库领域,并行计算作为一个新生并且强有力的事物,正在越来越多的方面发挥着重要的作用。

\section{OLAP相关}
在数据库领域,一个历史悠久的课题就是,对于大规模数据仓库,如何进行有效的管理,分析等一系列数据相关的操作。在这之中不得不提到的就是OLAP(On-Line Analytical Processing,线上分析处理)。这项技术将大量的原始数据聚合到特定的数据立方体中,在数据立方体中进行各种数据的分析,统计等等操作将比在原始数据集合中来得更加方便,快捷\cite{OLAP}。

在OLAP的基本概念中,数据立方体的构建是一项很基本的操作。其通过将海量的原始数据,按照它们之中的一些维度数据来进行划分,并且在此基础上进行聚合操作,从而得到一系列的分类聚合数据集合,用户可以在这些分类的数据聚合上更容易地寻找到自己所需要的对于原始数据集的整体,抑或是各个部分的概况。

在一个数据立方体(下文称cube)中,存在着许多种不同的聚合:它们都是对原始数据集的聚合,但是它们选择的聚合的维度有所不同。这样的聚合之中的某一个特定的聚合被称作cuboid。而在一个cuboid中,存在着许多cell,每一个cell中存放着在这个cuboid在给定的聚合方式(例如sum,count,histogram等)前提下,对同一类数据的聚合。这里的同类,指的是未被聚合的所有维度属性值都相同的那些数据。

对于OLAP的研究,同样开始于很早以前。1997年的一篇论文\cite{ChaudhuriSIGMOD1997}很早地从各个方面对于OLAP技术进行了一个全面的调研,并且指出了OLAP技术将来具体发展的形式。斯坦福大学的研究小组\cite{HarinarayanSIGMOD1996}则算是最早的具体研究如何高效快捷的生成OLAP的核心——数据立方体的论文之一。这篇论文首次提到“数据立方体中的cuboid并非每次都需要从原始数据生成”,并且基于这个观点给出了一套行之有效的预处理cuboid生成的评估方案,以及在此评估方案下的一套次最优的贪心算法。本文的其中一部分工作将基于他们的工作进行扩展与延伸。

近些年来,随着各种更新的系统架构,诸如分布式集群等的出现,OLAP从最开始的单机处理单机分析开始,渐渐打开了新的发展方向。2009年的一项研究成果\cite{PlattnerSIGMOD2009}则具有远见卓识地基于in-memory database的视点,对OLAP,以及OLTP,在它们运行的基本存储架构上进行了一系列的优化和创新,这也使得因此而诞生的SAP HANA\cite{SAPHANA}很快成为了包括in-memory database在内,目前世界上做的最优秀的in-memory platform之一。也有很多的文章着眼于并行计算的视点来考虑如何进行数据仓库相关一些操作的优化。俄亥俄州立大学的相关研究小组在2013年的论文中\cite{yuanVLDB2013},系统地给出了一套评估用并行硬件优化数据仓库相关操作的性能的标准。同一时期也有一系列文章则从不同的方面考虑了对于并行架构中的OLAP相关操作的优化。其中\cite{heVLDB2014}通过适当的控制在cache中的查询的调度从而使得CPU-GPU架构之间的通信与运行效率提高了,从而使与之相关的OLAP相关操作性能也得到了提升。另一个最近的研究成果\cite{KarnagelVLDB2017}则通过调整query候选,以及query中间结果在各个并行硬件单元上的分配,与\cite{heACMTDS2009}提出的评估模型不同,放弃了评估,转而采取即时地解决这些问题,从而使得其在OLAP的一些相关问题的处理上有了性能上的提升。

综上所述,OLAP是一个由来已久的问题,早些年的研究多是基于OLAP算法本身,而近些年来的研究更多专注于如何将其与现代的新兴系统架构,新兴硬件等等结合在一起,来让OLAP在现代技术的支持下展现出全新的活力。


\section{研究动机}
上述的研究背景中,我们可以发现:更多的研究要么专注于如何运用并行计算平台对并发/并行查询执行进行优化,要么专注于为在新的系统架构上部署高效的OLAP算法而设计具有针对性的软硬件结构的优化,似乎并没有一项研究来针对如何具体的运用并行架构体系来优化OLAP中数据立方体的生成算法本身。因此本研究就尝试着从这个领域着手,对这一问题提出自己的见解。
